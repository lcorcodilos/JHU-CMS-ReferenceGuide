% Introduction
\documentclass{article}
\usepackage{epigraph}
\usepackage{glossaries}
\usepackage{titlesec}
\titlespacing{\section}{0in}{0in}{0in}
\titlespacing{\subsection}{0in}{0in}{0in}
\usepackage{geometry}
\geometry{margin=1in}
\usepackage[utf8]{inputenc}
\usepackage[autostyle]{csquotes}
\setlength{\parindent}{0pt}
\setlength{\parskip}{1em}
\makeglossaries
\setglossarystyle{altlist}
% Glossary items 
% New entry template
% \newglossaryentry{refkey}{
%     name=tempname,
%     description={Description here}
% }

% DON'T PUT A PERIOD AT THE END OF YOUR DESCRIPTIONS. MAKEGLOSSARIES DOES IT AUTOMATICALLY (though I don't know why...)

\newglossaryentry{Nminus1}{
    name=N-1 Plot,
    description={A plot as a function of variable X where a selection has been applied to all variables except X (N total variables with N-1 cuts applied). These are typically used to either compare the shapes of signal and background as a function of X or to scan for an optimal point to place a cut to maximize the (cumulative) $S/\sqrt(B)$. It's not uncommon to also do N-2 or N-3 plots depending on the scenario}
}
 
\newglossaryentry{signalstrength}{
    name=Signal strength/$r$/$\mu$,
    description={A normalization factor that is fit for when comparing data against a background estimate. In the backgroud-only hypothesis, this is 0 because the hypothesis assumes no signal exists. In the so-called signal+background hypothesis, the signal strenght is left to float and ``fill-in'' any peaks in the distribution. If the signal simulation template is normalized to its theoretical cross section and the luminosity of the data being analyzed, then a signal strenght of one means the template is exact. A value of two means there is twice as much signal as the simulation (including the cross section value) predicts. If the signal simulation is only scaled to the luminosity (this means the cross section is effectively set to 1), then fitting for the signal strength is equivalent to fitting for the true cross section}
}
\glsaddallunused
\begin{document}

\title{The Official Unoffical CMS Reference Guide for Graduate Students}
\author{Oz Amram, Lucas Corcodilos, Cristina Mantilla Suarez,\\ and other JHU graduate students}

\maketitle

\epigraph{That's one of those pieces of information in the \textit{Book of Everything You Need to Know} that nobody wrote.}{A senior graduate student}

The introductory quote (or some form of it) has been said too many times by students who know enough to fill a book but are too busy to write it. This project is meant to remedy that issue for physics at CMS. There are simply too many pieces of the collider, detectors, software, and beaurecratic structure for any one person to know everything. The success of the experiments and the analyses they perform rely on every contributor's ability to be an expert on a few things, to share their expertise, and to learn about areas of inexperience from other experts when necessary. 

One of the greatest difficulties of the final responsibility is finding the proper documentation. Even if a group within CMS does an excellent job of documenting their procedures and providing easy-to-use tools, there may not be an obvious way to look for their resources! While searching the CMS TWiki or Google is certainly an option, it typically provides search results totally unrelated to the topic of interest and it can be hard to decipher which pages are recent and which are out-of-date (not to mention all of the personal pages with wrong information!). 

This guide attempts to remedy this issue by crowd sourcing references, links, explanations, and sub-guides from the experts in one searchable document. Whether you are enitrely new and need an encyclopedia to reference or are an old hand that just needs to learn about the latest lepton scale factors, there's information for everyone. And if the information you need isn't here, go look for it and become the new resident expert by contributing what you've learned to this document! 

\section*{Sharing is caring}
Sharing knowledge is the key to this project's success so please consider contributing what you know. There's no need to write pages all at once. Just keep this document in mind after you've discovered a nuianced issue with triggers or write an email to an undergraduate describing jet reclustering. Those are excellent moments to contribute because the information is fresh in your mind and most of the work of writing it out may already be done. Simply edit the LaTeX yourself or just drop the text into an issue on GitHub and let others do it for you.

Enjoy and feel free to ask questions!

\clearpage

\section{Corrections, Weights, and Scaling}
This section covers different types of corrections, weights, re-weights, scaling, and other multiplicative factors required because of either differences in simulation and data or known inconsistencies in reconstruction algorithms that affect objects in both simulation and data. Each section briefly covers the reason for the corrections, how they are derived, and where to find the latest values and uncertainties.

\subsection{Jet Corrections}

\subsection{Pileup}

\subsection{Tagging}

\subsection{Triggers}
\section{Accessing Data and Monte Carlo Sets}

\subsection*{Using DAS (Data Aggregation System)}

The easiest way to look for MC and data samples is with DAS. Which is \href{https://cmsweb.cern.ch/das/}{here}. 
The datasets will generally have the format ``/XXXX/YYYY/ZZZZ`` .

The XXXX will be something about the actual physics content of the dataset, eg WW\_TuneCP5\_13TeV-pythia8 is for WW simulation made with Pythia using the tune CP5. 
For data it will describe the triggers used for that dataset eg ``SingleMuon``.

The YYYY will be about the production info of that data set. 
For MC it will be something like ``RunIIAutumn18MiniAOD-102X\_upgrade2018\_realistic\_v15-v2`` which means it was made during the RunIIAutumn18 round of MC production 
in the 102X release with some other info. For data it will be something like Run2018A-17Sep2018-v2 which means its from RunA from 2018 data taking and it was reconstructed in the 17Sep2018 batch.
Often times data is reconstructed multiple (Re-Reco'ed) times with progressively better calibrations, usually you want the latest one. 

The ZZZZ will be the data type like MINIAOD or NANOAOD. 

You can use wildcards to help you find things if you don't know the exact name. 
Once you find a dataset you can look at a list of all its files  (its often a good idea to run on one file before you run a full crab job).
There is also a link to cross-section database entry for that dataset which sometimes works. Usually this cross section is not the most accurate one, but it is useful as a starting point. 


\subsection*{$t\bar{t}$ Simulation}
\begin{itemize}
    \item For 2016, $t\bar{t}$ MC was generated for inclusive decays (meaning, all possible branchings of the two top quarks were allowed). For 2017 and 2018, decays were split into all-hadronic, semi-leptonic, and leptonic (denoted 2L2Nu).
\end{itemize}


% Clean page and print
\clearpage
\printglossary
\end{document}

