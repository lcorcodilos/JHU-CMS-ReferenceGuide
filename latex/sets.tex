\section{Accessing Data and Monte Carlo Sets}

\subsection*{Using DAS (Data Aggregation System)}

The easiest way to look for MC and data samples is with DAS. Which is \href{https://cmsweb.cern.ch/das/}{here}. 
The datasets will generally have the format ``/XXXX/YYYY/ZZZZ`` .

The XXXX will be something about the actual physics content of the dataset, eg WW\_TuneCP5\_13TeV-pythia8 is for WW simulation made with Pythia using the tune CP5. 
For data it will describe the triggers used for that dataset eg ``SingleMuon``.

The YYYY will be about the production info of that data set. 
For MC it will be something like ``RunIIAutumn18MiniAOD-102X\_upgrade2018\_realistic\_v15-v2`` which means it was made during the RunIIAutumn18 round of MC production 
in the 102X release with some other info. For data it will be something like Run2018A-17Sep2018-v2 which means its from RunA from 2018 data taking and it was reconstructed in the 17Sep2018 batch.
Often times data is reconstructed multiple (Re-Reco'ed) times with progressively better calibrations, usually you want the latest one. 

The ZZZZ will be the data type like MINIAOD or NANOAOD. 

You can use wildcards to help you find things if you don't know the exact name. 
Once you find a dataset you can look at a list of all its files  (its often a good idea to run on one file before you run a full crab job).
There is also a link to cross-section database entry for that dataset which sometimes works. Usually this cross section is not the most accurate one, but it is useful as a starting point. 


\subsection*{$t\bar{t}$ Simulation}
\begin{itemize}
    \item For 2016, $t\bar{t}$ MC was generated for inclusive decays (meaning, all possible branchings of the two top quarks were allowed). For 2017 and 2018, decays were split into all-hadronic, semi-leptonic, and leptonic (denoted 2L2Nu).
\end{itemize}
